\documentclass[a4paper,12pt]{extarticle}

\usepackage[T2A]{fontenc}
\usepackage[utf8x]{inputenc}
\usepackage[english,russian]{babel}
\setlength{\parindent}{0pt}

\usepackage{microtype}
\usepackage{amsmath}
%\usepackage{amssymb}
%\usepackage{calligra}
  %\DeclareFontFamily{U}{calligra}{}
  %\DeclareFontShape{U}{calligra}{m}{n}{<->s*[1.24]callig15}{}
\usepackage[left=1.6cm,right=1.6cm,top=1.4cm,bottom=1cm]{geometry}
\usepackage{xspace}
\usepackage{siunitx}
  \sisetup{exponent-product=\cdot,
           output-decimal-marker = {,}}

\usepackage{upgreek}
\usepackage{graphicx}
  \graphicspath{{/home/slisakov/yadisk/plots/}}
\usepackage{floatrow}
\usepackage{caption}
  \captionsetup{labelformat=empty}
\usepackage{wrapfig}
  \setlength{\intextsep}{0pt}

\usepackage{array}
  \newcolumntype{L}[1]{>{\raggedright\let\newline\\\arraybackslash\hspace{0pt}}p{#1}}
  \newcolumntype{C}[1]{>{\centering\let\newline\\\arraybackslash\hspace{0pt}}p{#1}}
  \newcolumntype{R}[1]{>{\raggedleft\let\newline\\\arraybackslash\hspace{0pt}}p{#1}}


\usepackage[auto-label]{exsheets}
  %\renewcommand\thequestion{\thechapter\arabic{question}}
  \SetupExSheets{headings=runin, counter-format = qu}
  \DeclareInstance{exsheets-heading}{runin}{default}{
    runin = true,
    number-post-code = \bf.\hspace{0.3em},
    attach = { main[hc,vc]number[l,vc](-2em,0pt) }
  }

\pagestyle{empty}
\input{/home/slisakov/yadisk/documents/1514/physics/10/pb_10/abbr_10.tex}
\input{/home/slisakov/yadisk/documents/1514/physics/9/pb_9/abbr.tex}
\def\arraystretch{1.3}

\def\dem[#1,#2,#3]{\texttt{(20#1–В#2-#3)}\hspace{3pt}\xspace}
%\def\sg[#1]{\texttt{(#1)}\hspace{3pt}\xspace}
\def\sg[#1,#2]{\texttt{(№#1,#2)}\hspace{3pt}\xspace}
\def\dva{\textbf{\textit{два}}\xspace}
\def\vse{\textbf{\textit{все}}\xspace}
\def\bu[#1]{\textbf{\uline{#1}}}
\def\bi[#1]{\textbf{\textit{#1}}}
\def\kuda{\textbf{\textit{(вверх, вниз, влево, вправо, от наблюдателя, к наблюдателю)}}\xspace}
\def\podberite{%
  К каждой позиции первого столбца подберите соответствующую позицию
  второго и запишите в таблицу выбранные цифры под соответствующими
  буквами.\\}
\def\izm{%
  \vspace*{1mm}Для каждой величины определите соответствующий характер изменения:
  \begin{enumerate}
    \item увеличивается
    \item уменьшается
    \item не изменяется
  \end{enumerate}
  Запишите в таблицу выбранные цифры для каждой физической
  величины. Цифры в ответе могут повторяться. \\
}


\usepackage[export]{adjustbox}
%%% Instead of wrapfig, for 1 paragraph
\newsavebox{\illustration}% used to pass image
\newenvironment{leftfigure}[1]{% #1=line number for start of illustration (1=first)
  \setlength{\dimen2}{\dimexpr \wd\illustration+\columnsep}% indentation
  \setlength{\dimen0}{\dimexpr \linewidth-\dimen2}% reduced line width
  \setlength{\dimen1}{\dimexpr \ht\illustration + \dp\illustration}% total height
  \setlength{\dimen3}{0pt}% image offset
  \commonfigure{#1}}{}
\newenvironment{rightfigure}[1]{% #1=line number for start of illustration (1=first)
  \setlength{\dimen3}{\dimexpr \linewidth-\wd\illustration}% offset to image
  \setlength{\dimen0}{\dimexpr \dimen3-\columnsep}% reduced line width
  \setlength{\dimen1}{\dimexpr \ht\illustration + \dp\illustration}% total height
  \setlength{\dimen2}{0pt}% indentation
  \commonfigure{#1}}{}
\newcommand{\commonfigure}[1]{% combine environments
  \count1=\numexpr \dimen1/\baselineskip\relax
  \ifdim \dimen1>\the\count1\baselineskip\relax
    \advance\count1 by 1
  \fi
  \count2=0% construct \parshape arguments
  \ifnum #1<2\relax
    \edef\shape{\the\numexpr \count1+1}%
  \else
    \edef\shape{\the\numexpr \count1+#1}%
    \loop\advance\count2 by 1
    \ifnum\count2<#1\relax
      \edef\shape{\shape\space 0pt \the\linewidth}%
    \repeat
    \count2=0
  \fi
  \loop\ifnum\count2<\count1
    \edef\shape{\shape\space \the\dimen2\space \the\dimen0}%
    \advance\count2 by 1
  \repeat
  \edef\shape{\shape\space 0pt \the\linewidth}%
  \par\noindent\rlap{\hspace{\dimen3}% overlap illustration
    \raisebox{\dimexpr \ht\strutbox+\baselineskip-#1\baselineskip-\ht\illustration}%
      [0pt][0pt]{\usebox\illustration}}%
  \strut\vspace{-3.1ex}% approximate the vertical spacing for question
  \everypar{\parshape=\shape\everypar{}}%
}



\begin{document}

\section*{Реальные задания ЕГЭ–2020 по физике}

%\setcounter{question}{0}
%\begin{question}[ID=1]
    %Дан график $x(t)$, парабола с вершиной в нуле, проходит через (2;2).
    %Определите проекцию $a_x$ ускорения тела.
  %\begin{solution} $a_x=1\mss$ \end{solution}
%\end{question}

\setcounter{question}{26}
\begin{question}[ID=27]
    Цепь, изображённая на рисунке, состоит из источника с ненулевым
    внутренним сопротивлением, двух одинаковых резисторов,
    конденсатора и двух идеальных вольтметров. Как изменятся
    показания вольтметров, если ключ замкнуть? Ответ объясните,
    опираясь на известные вам законы физики.

    \hfil\includegraphics[width=0.5\textwidth]{27_2020_real}\\
  \begin{solution} \end{solution}
\end{question}


\begin{question}[ID=28]
  Груз медленно поднимают с помощью рычага, прикладывая силу \FN[350] как
  показано на рисунке. Рычаг сделан из однородного
  стержня массой \mkg[10] и длиной \Lm[4]. К рычагу на расстоянии $b$ от
  шарнира прикреплён груз массой \Mkg[75]. Найти расстояние $b$. \\[1ex]

    \hfil\includegraphics[width=0.3\textwidth]{28v17_2020}\\
  \begin{solution} $b = \dfrac{FL - mgL/2}{Mg}$ \end{solution}
\end{question}

\begin{question}[ID=29]
  Грузы массой \mkg[0.5] и $M$ соединены невесомой нерастяжимой нитью,
  перекинутой через блок как показано на рисунке. Наклон плоскости, на
  которой расположен груз массой $M$, составляет \agr[30]. Коэффициент
  трения между грузом и плоскостью \mub[0.3]. Найти максимальную массу
  груза $M$, при которой система ещё находится в покое.
  Трением в блоке пренебречь.

    \hfil\includegraphics[width=0.4\textwidth]{29_2020_real}\\
  \begin{solution} $M=\dfrac{m}{\sina-\mu\cosa} \approx \num{2.1}\kg$ \end{solution}
\end{question}

\begin{question}[ID=30]
  Углекислый газ $\text{CO}_2$ находится в сосуде при температуре 300\K.
  При нагревании газа он разлагается в соответствии со следующим
  уравнением: $2\,\text{CO}_2 \leftrightarrow 2\,\text{CO} + \text{O}_2$.
  График зависимости количества распавшихся молекул $\text{CO}_2$ изображён
  на рисунке. Найдите парциальное давление кислорода, если давление смеси
  газов при температуре 3000\K составляет 100\kPa.

    \hfil\includegraphics[width=0.4\textwidth]{30_2020_real}\\
  \begin{solution} $p_{\text{O}_2} = \dfrac{3p}{13} \approx 23\kPa$ \end{solution}
\end{question}

\begin{question}[ID=31]
  Круглый алюминиевый виток находится в магнитном поле с индукцией
  \num{5e-3}\;Тл, причём вектор индукции магнитного поля перпендикулярен
  плоскости витка. Виток растягивают вдоль диаметра, при этом по нему
  протекает заряд $q=4\mCl$. Найдите радиус витка, если отношение
  сопротивления витка к его длине равно $\rho_l=\num{0.1}$\;Ом/м.

  \begin{solution} $r = \dfrac{2q}{B}\rho_l = \num{0.16}\m$ \end{solution}
\end{question}

\begin{question}[ID=32]
  С помощью тонкой собирающей линзы с фокусным расстоянием 15\cm на экране
  получено чёткое изображение предмета с пятикратным увеличением. Затем
  экран придвигают к линзе на 30\cm, и предмет двигают так, чтобы снова
  получилось чёткое изображение на экране. Линзу при этом не перемещали.
  Найдите новое увеличение. Постройте на рисунке изображения, полученные в
  линзе.
  \begin{solution} $\Upgamma_2 = 3$ \end{solution}
\end{question}

%\newpage\twocolumn
%\printsolutions
\end{document}
